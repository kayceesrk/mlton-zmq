\chap{The basis library}{basis-library}

The basis library is implemented with about 12,000 lines of SML code.  There is
roughly one file for each signature and structure that the library specification
defines.  The files are grouped in directories in the same way that the
corresponding modules are grouped in the basis library documentation.  Here is
an overview of the {\tt basis-library} directory.

\begin{description}
\place{arrays-and-vectors general integer io list posix real system text}
SML code for basis library modules.

\place{basis.sml}
Automatically constructed by {\tt bin/check-basis}.  Used to type check the
basis libary under {\smlnj}.

\place{bind-basis}
A list of the files (in order) that define what is exported by the basis
library.

\place{build-basis}
A list of the files (in order) used to construct the basis library.

\place{Makefile}
Only has a target to clean the directory.

\place{misc}
SML code that didn't fit anywhere else.  In particular, the {\tt Primitive}
structure.

\place{mlton}
The {\tt MLton} structure, which is not part of the standard basis library.
For more details on what {\tt MLton} provides, see the {\userguide}.

\place{sml-nj}
The {\tt SMLofNJ} and {\tt Unsafe} structures, which are not part of the
standard basis library.

\place{top-level}
Files describing the overloads, infixes, modules, types, and values that the
basis library makes available to user programs.
\end{description}

\subsec{How {\mlton} builds the basis environment}{build-basis-env}
The {\tt forceBasisLibrary} function in \code{\tt mlton/main/compile.sml} builds
the basis environment that is used to compile user programs.  Conceptually, the
basis environment is constructed in two steps.  First, all of the files in {\tt
build-basis} are concatenated together and evaluated to produce an environment
$E$.  Then, all of the files in {\tt bind-basis} are concatenated and evaluated
in environment $E$ to produce a new environment $E'$, which is the top-level
environment.  Another way to view it is that every user program is prefixed by
the following.
\begin{verbatim}
local
  <concatenate files in build-basis>
in
  <concatenate files in bind-basis>
end
\end{verbatim}
This view is not strictly accurate because some of the files are not SML (they
use the {\tt \_prim}, {\tt \_ffi}, and {\tt \_overload} syntaxes) and because SML
does not allow local functor or signature declarations.  Here is a description
of the basis files that are not SML.
\begin{description}
\place{misc/primitive.sml}
Defines the {\tt Primitive} structure, which binds (via the {\tt \_prim}
syntax) all of the primitives provided by the compiler that the basis library
uses.
\place{mlton/syslog.sml}
Defines constants and FFI routines used to implement {\tt MLton.Syslog}.
\place{posix/primitive.sml}
Defines the {\tt PosixPrimitive} structrue, which binds the constants and FFI
routines used to implement the {\tt Posix} structure.
\place{top-level/overloads.sml}
Defines the overloaded variables available at the top-level the {\tt \_overload}
syntax: {\tt \_overload $x$: $ty$ as $y_0$ and $y_1$ and ...}
\end{description}

\subsection{Modifying the basis library}

If you modify the basis library, you should first check that your modifications
are type correct using the {\tt bin/check-basis} script.  Since this {\mlton}
does not have a proper typechecker, this script uses {\smlnj}.  First, it
concatenates the files as described in \secref{build-basis-env} into one file,
{\tt basis.sml}.  It also replaces the nonstandard syntax ({\tt \_prim}, etc.) 
and declares the toplevel types to match {\mlton}'s (necessary since {\smlnj}
uses 31 bits while {\mlton} uses 32).  It then feeds {\tt basis.sml} to
{\smlnj}.  If there are no type errors, a message like the following will
appear.
\begin{verbatim}
stdIn:12213.1-12213.14 Error: operator is not a function [tycon mismatch]
  operator: unit
  in expression:
    () ()
\end{verbatim}
This error message is intentionally introduced by {\tt check-basis} at the end
of {\tt basis.sml} to make it clear that {\smlnj} reached the end of {\tt
basis.sml} and has hence type checked the entire basis.

Once you have a basis library that type checks, you need to create a new version
of {\mlton} that uses this library.  {\mlton} preprocess the basis library to
create a {\tt world.mlton} file that contains the basis environment.  The {\tt
world.mlton} file is stored in the {\tt lib} directory and is loaded by {\tt
mlton} when compiling a user program (see the {\tt bin/mlton} script).  To build
a new {\tt world.mlton}, run {\tt make world} from within the sources directory.

\subsection{The {\tt misc} directory}

\begin{description}

\place{cleaner.sig}
Functions for register ``cleaning'' functions to be run at certain times, in
particular at program exit.  The {\tt TextIO} module uses these cleaners to
ensure that IO buffers are flushed upon exit.

\place{suffix.sml}
Code that is (conceptually) concatenated on to the end of every user program.
It just calls {\tt OS.Process.exit}.  The {\tt forceBasisLibrary} function
ensures that {\tt suffix.sml} is elaborated in an environment where the basis
library {\tt OS} structure is available.

\place{top-level-handler.sml}
This defines the top level exception handler that is installed (via a special
compiler primitive) in the basis library, before any user code is run.

\end{description}

\subsection{Dead-code elimination}

In order to compile small programs rapidly and to cut down on executable size,
{\tt mlton} runs a pass of dead-code elimination ({\tt
mlton/core-ml/dead-code.sig}) to eliminate as much of the basis library as
possible.  The dead-code elimination algorithm used is not safe in general, and
only works because the basis library implementation has special properties:
\begin{itemize}
\item it terminates
\item it performs no I/O
\item it doesn't side-effect top-level variables
\end{itemize}
The dead code elimination simply includes the minimal set of
declarations from the basis so that their are no free variables in the
user program (or basis).  Hence, if you do something like the
following in the basis, it will break.
\begin{verbatim}
val r = ref 13
val _ = r := 14
\end{verbatim}
The dead code elimination will remove the {\tt val \_ = ...} binding.
